\hypertarget{README_8md_source}{\section{R\-E\-A\-D\-M\-E.\-md}
}

\begin{DoxyCode}
00001 \textcolor{preprocessor}{# README File for the Mimetic Methods Toolkit (MTK)}
00002 \textcolor{preprocessor}{}
00003 By: **Eduardo J. Sanchez, Ph.D. - esanchez at mail dot sdsu dot edu**
00004     \_\_\_\_\_\_\_\_\_\_\_\_\_\_\_\_\_\_\_\_\_\_\_\_\_\_\_\_\_\_\_\_\_\_\_\_\_\_\_\_\_\_\_\_\_\_\_\_\_\_\_\_\_\_\_\_\_\_\_\_\_\_\_\_\_\_
00005 
00006 \textcolor{preprocessor}{## 1. Description}
00007 \textcolor{preprocessor}{}
00008 We define numerical methods that are based on discretizations preserving the
00009 properties of their continuum counterparts to be **mimetic**.
00010 
00011 The **Mimetic Methods Toolkit (MTK)** is a C++ library \textcolor{keywordflow}{for} mimetic numerical
00012 methods. It is arranged as a set of classes \textcolor{keywordflow}{for} **mimetic quadratures**,
00013 **mimetic interpolation**, and **mimetic discretization** methods \textcolor{keywordflow}{for} the
00014 numerical solution of ordinary and partial differential equations.
00015 
00016 This collection of **MATLAB Wrappers \textcolor{keywordflow}{for} the MTK (MMTK)** allows developers to
00017 invoke the MTK from a MATLAB environment.
00018 
00019     \_\_\_\_\_\_\_\_\_\_\_\_\_\_\_\_\_\_\_\_\_\_\_\_\_\_\_\_\_\_\_\_\_\_\_\_\_\_\_\_\_\_\_\_\_\_\_\_\_\_\_\_\_\_\_\_\_\_\_\_\_\_\_\_\_\_
00020 
00021 \textcolor{preprocessor}{## 2. Dependencies}
00022 \textcolor{preprocessor}{}
00023 This README assumes \hyperlink{Div1D_8m_a4603254d9990f7140c024d51302d1a8f}{all} of these dependencies are installed in the following
00024 folder:
00025 
00026 ```
00027 $(HOME)/Libraries/
00028 ```
00029 
00030 In \textcolor{keyword}{this} version, the MTK optionally uses ATLAS-optimized BLAS and LAPACK
00031 routines \textcolor{keywordflow}{for} the \textcolor{keyword}{internal} computation on some of the layers. However, ATLAS
00032 requires both BLAS and LAPACK in order to create their optimized distributions.
00033 Therefore, the following dependencies tree arises:
00034 
00035 ### For Linux and OS X:
00036 
00037 1. MATLAB R2014a or greater - Available from: http:\textcolor{comment}{//www.mathworks.com/}
00038 
00039 1. MTK - Available from: http:\textcolor{comment}{//csrc.sdsu.edu/mtk/}
00040     \_\_\_\_\_\_\_\_\_\_\_\_\_\_\_\_\_\_\_\_\_\_\_\_\_\_\_\_\_\_\_\_\_\_\_\_\_\_\_\_\_\_\_\_\_\_\_\_\_\_\_\_\_\_\_\_\_\_\_\_\_\_\_\_\_\_
00041 
00042 ## 3. Installation
00043 
00044 The following instructions assume MATLAB R2014a or greater.
00045 
00046 You have two options, either follow the instructions given, **expert install**,
00047 or use the provided **patch file**, **naive install**.
00048 
00049 ### EXPERT PART 1. CONFIGURATION OF THE MEX COMPILER.
00050 
00051 From your MATLAB prompt, type:
00052 
00053 ```
00054 >> mex -setup C++
00055 MEX configured to use \textcolor{stringliteral}{'g++'} \textcolor{keywordflow}{for} C++ language compilation.
00056 Warning: The MATLAB C and Fortran API has changed to support MATLAB
00057      variables with more than 2^32-1 elements. In the near future
00058      you will be required to update your code to utilize the
00059      \textcolor{keyword}{new} API. You can find more information about \textcolor{keyword}{this} at:
00060 
00061 http:\textcolor{comment}{//www.mathworks.com/help/matlab/matlab\_external/upgrading-mex-files-to-use-}
00062 64-bit-api.html.
00063 >>
00064 ```
00065 
00066 ### EXPERT PART 2. CONFIGURATION OF THE MEX COMPILER SETUP FILE.
00067 
00068 The previous step creates the following file:
00069 
00070 ```
00071 $(HOME)/.matlab/R2014a/mex\_C++\_glnxa64.xml
00072 ```
00073 
00074 The purpose of \textcolor{keyword}{this} section is to configure the MATLAB R2014a Mex compiler so
00075 that it can work with the latest C++ standard (C++11).
00076 
00077 Please execute the following changes on the aforementioned file (line numbers
00078 may differ):
00079 
00080 Lines 26 and 27:
00081 
00082 ```
00083 CMDLINE1=\textcolor{stringliteral}{"$CXX -std=c++11 -c $DEFINES $INCLUDE $CXXFLAGS $OPTIM $SRC -o $OBJ"}
00084 CMDLINE2=\textcolor{stringliteral}{"$LDXX -std=c++11 $LDFLAGS $LDTYPE $LINKOPTIM $LINKEXPORT $OBJS $CXXLIBS $LINKLIBS -o $EXE"}
00085 ```
00086 
00087 Line 33:
00088 
00089 ```
00090 CXXFLAGS=\textcolor{stringliteral}{"-std=c++11 -ansi -fexceptions -fPIC -fno-omit-frame-pointer -pthread"}
00091 ```
00092 
00093 Line 35 and 26:
00094 
00095 ```
00096 CXXOPTIMFLAGS=\textcolor{stringliteral}{"-std=c++11 -O -DNDEBUG"}
00097 CXXDEBUGFLAGS=\textcolor{stringliteral}{"-std=c++11 -g"}
00098 ```
00099 
00100 Line 38:
00101 
00102 ```
00103 LDXX=\textcolor{stringliteral}{"gfortran"}
00104 ```
00105 
00106 Line 59:
00107 
00108 ```
00109 CXXFLAGS=\textcolor{stringliteral}{"-std=c++11 -ansi -pthread"}
00110 ```
00111 
00112 ### NAIVE PART 1. CONFIGURATION OF THE MEX COMPILER.
00113 
00114 From your MATLAB prompt, type:
00115 
00116 ```
00117 >> mex -setup C++
00118 MEX configured to use \textcolor{stringliteral}{'g++'} for C++ language compilation.
00119 Warning: The MATLAB C and Fortran API has changed to support MATLAB
00120      variables with more than 2^32-1 elements. In the near future
00121      you will be required to update your code to utilize the
00122      new API. You can find more information about this at:
00123 
00124 http:\textcolor{comment}{//www.mathworks.com/help/matlab/matlab\_external/upgrading-mex-files-to-use-}
00125 64-bit-api.html.
00126 >>
00127 ```
00128 
00129 ### NAIVE PART 2. CONFIGURATION OF THE MEX COMPILER SETUP FILE.
00130 
00131 ```
00132 cd $HOME/.matlab/R2014a
00133 chmod +w mex\_C++\_glnxa64.xml
00134 patch < mex\_C++\_glnxa64.patch
00135 chmod -w mex\_C++\_glnxa64.xml
00136 ```
00137 Exit terminal, and restart MATLAB. You can use C++11 to create MEX files now!
00138 
00139 ### PART 3: CONFIGURATION OF THE MAKEFILE.
00140 
00141 The following steps are required the build and test the MTK. Please use the
00142 accompanying `Makefile.inc` file, which should provide a solid template to
00143 start with. The following command provides help on the options for make:
00144 
00145 ```
00146 $ make help
00147 -----
00148 Makefile for the MMTK.
00149 
00150 Options are:
00151 - \hyperlink{Div1D_8m_a4603254d9990f7140c024d51302d1a8f}{all}: builds he library, the tests, and examples.
00152 
00153 - gendoc: generates the documentation for the library.
00154 
00155 - clean: cleans ALL the generated files.
00156 -----
00157 ```
00158 
00159 ### PART 4. BUILD THE MMTK.
00160 
00161 From your shell, at the base folder of the MMTK, just type:
00162 
00163 ```
00164 make
00165 ```
00166 
00167 If successful you\textcolor{stringliteral}{'ll read:}
00168 \textcolor{stringliteral}{}
00169 \textcolor{stringliteral}{```}
00170 \textcolor{stringliteral}{----- Library created! Check in /home/ejspeiro/Dropbox/MTK/lib}
00171 \textcolor{stringliteral}{```}
00172 \textcolor{stringliteral}{    \_\_\_\_\_\_\_\_\_\_\_\_\_\_\_\_\_\_\_\_\_\_\_\_\_\_\_\_\_\_\_\_\_\_\_\_\_\_\_\_\_\_\_\_\_\_\_\_\_\_\_\_\_\_\_\_\_\_\_\_\_\_\_\_\_\_}
00173 \textcolor{stringliteral}{}
00174 \textcolor{stringliteral}{## 4. Frequently Asked Questions}
00175 \textcolor{stringliteral}{}
00176 \textcolor{stringliteral}{Q: Why haven'}t you guys implemented GBS to build the library?
00177 A: I\textcolor{stringliteral}{'m on it as we speak! ;)}
00178 \textcolor{stringliteral}{}
00179 \textcolor{stringliteral}{Q: When will the other flavors be ready?}
00180 \textcolor{stringliteral}{A: Soon! I'}m working on getting help on developing those.
00181 
00182 Q: Is there any main reference when it comes to the theory on Mimetic Methods?
00183 A: Yes! Check: http:\textcolor{comment}{//www.csrc.sdsu.edu/mimetic-book}
00184 
00185 Q: Do I need to generate the documentation myself?
00186 A: You can if you want to... but if you DO NOT want to, just go to our website.
00187     \_\_\_\_\_\_\_\_\_\_\_\_\_\_\_\_\_\_\_\_\_\_\_\_\_\_\_\_\_\_\_\_\_\_\_\_\_\_\_\_\_\_\_\_\_\_\_\_\_\_\_\_\_\_\_\_\_\_\_\_\_\_\_\_\_\_
00188 
00189 ## 5. Contact, Support, and Credits
00190 
00191 The MTK is developed by researchers and adjuncts to the
00192 [Computational Science Research Center (CSRC)](http:\textcolor{comment}{//www.csrc.sdsu.edu/)}
00193 at [San Diego State University (SDSU)](http:\textcolor{comment}{//www.sdsu.edu/).}
00194 
00195 Developers are members of:
00196 
00197 1. Mimetic Numerical Methods Research and Development Group.
00198 2. Computational Geoscience Research and Development Group.
00199 3. Ocean Modeling Research and Development Group.
00200 
00201 Currently the developers are:
00202 
00203 - **Eduardo J. Sanchez, Ph.D. - esanchez at mail dot sdsu dot edu** - @ejspeiro
00204 - Jose E. Castillo, Ph.D. - jcastillo at mail dot sdsu dot edu
00205 - Guillermo F. Miranda, Ph.D. - unigrav at hotmail dot com
00206 - Christopher P. Paolini, Ph.D. - paolini at engineering dot sdsu dot edu
00207 - Angel Boada.
00208 - Johnny Corbino.
00209 - Raul Vargas-Navarro.
00210 
00211 Finally, please feel free to contact me with suggestions or corrections:
00212 
00213 **Eduardo J. Sanchez, Ph.D. - esanchez at mail dot sdsu dot edu** - @ejspeiro
00214 
00215 Thanks and happy coding!
\end{DoxyCode}
