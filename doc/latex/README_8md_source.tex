\hypertarget{README_8md_source}{\section{R\-E\-A\-D\-M\-E.\-md}
}

\begin{DoxyCode}
00001 \textcolor{preprocessor}{# MATLAB wrappers for the Mimetic Methods Toolkit (MMTK)}
00002 \textcolor{preprocessor}{}
00003 By: **Eduardo J. Sanchez, Ph.D. - esanchez at mail dot sdsu dot edu**
00004     \_\_\_\_\_\_\_\_\_\_\_\_\_\_\_\_\_\_\_\_\_\_\_\_\_\_\_\_\_\_\_\_\_\_\_\_\_\_\_\_\_\_\_\_\_\_\_\_\_\_\_\_\_\_\_\_\_\_\_\_\_\_\_\_\_\_
00005 
00006 \textcolor{preprocessor}{## 1. Description}
00007 \textcolor{preprocessor}{}
00008 We define numerical methods that are based on discretizations preserving the
00009 properties of their continuum counterparts to be **mimetic**.
00010 
00011 The **Mimetic Methods Toolkit (MTK)** is a C++ library \textcolor{keywordflow}{for} mimetic numerical
00012 methods. It is arranged as a set of classes \textcolor{keywordflow}{for} **mimetic quadratures**,
00013 **mimetic interpolation**, and **mimetic discretization** methods \textcolor{keywordflow}{for} the
00014 numerical solution of ordinary and partial differential equations. The MTK can
00015 be found in:
00016 
00017 [The Mimetic Methods Toolkit (MTK)](https:\textcolor{comment}{//github.com/ejspeiro/MTK)}
00018 
00019 This collection of **MATLAB Wrappers \textcolor{keywordflow}{for} the MTK (MMTK)** allows developers to
00020 invoke the MTK from a MATLAB environment.
00021 
00022 Finally, a collection of grid visualization routines that is compatible with
00023 these wrappers can be found in:
00024 
00025 [MATLAB Visualizers \textcolor{keywordflow}{for} Uniform Staggered Grids](https:\textcolor{comment}{//github.com/ejspeiro/UniStgGrid-Visualizers)}
00026     \_\_\_\_\_\_\_\_\_\_\_\_\_\_\_\_\_\_\_\_\_\_\_\_\_\_\_\_\_\_\_\_\_\_\_\_\_\_\_\_\_\_\_\_\_\_\_\_\_\_\_\_\_\_\_\_\_\_\_\_\_\_\_\_\_\_
00027 
00028 \textcolor{preprocessor}{## 2. Dependencies}
00029 \textcolor{preprocessor}{}
00030 This README assumes \hyperlink{Div1D_8m_a4603254d9990f7140c024d51302d1a8f}{all} of these dependencies are installed in the following
00031 folder:
00032 
00033 ```
00034 $(HOME)/Libraries/
00035 ```
00036 
00037 In \textcolor{keyword}{this} version, the MTK optionally uses ATLAS-optimized BLAS and LAPACK
00038 routines \textcolor{keywordflow}{for} the \textcolor{keyword}{internal} computation on some of the layers. However, ATLAS
00039 requires both BLAS and LAPACK in order to create their optimized distributions.
00040 Therefore, the following dependencies tree arises:
00041 
00042 ### For Linux and OS X:
00043 
00044 1. MATLAB R2014a or greater - Available from: http:\textcolor{comment}{//www.mathworks.com/}
00045 
00046 1. MTK - Available from: http:\textcolor{comment}{//csrc.sdsu.edu/mtk/}
00047 
00048 Plus \hyperlink{Div1D_8m_a4603254d9990f7140c024d51302d1a8f}{all} of the dependencies the MTK entails.
00049     \_\_\_\_\_\_\_\_\_\_\_\_\_\_\_\_\_\_\_\_\_\_\_\_\_\_\_\_\_\_\_\_\_\_\_\_\_\_\_\_\_\_\_\_\_\_\_\_\_\_\_\_\_\_\_\_\_\_\_\_\_\_\_\_\_\_
00050 
00051 ## 3. Installation
00052 
00053 The following instructions assume MATLAB R2014a or greater.
00054 
00055 You have two options, either follow the instructions given, **expert install**,
00056 or use the provided **patch file**, **naive install**.
00057 
00058 ### EXPERT PART 1. CONFIGURATION OF THE MEX COMPILER.
00059 
00060 From your MATLAB prompt, type:
00061 
00062 ```
00063 >> mex -setup C++
00064 MEX configured to use \textcolor{stringliteral}{'g++'} \textcolor{keywordflow}{for} C++ language compilation.
00065 Warning: The MATLAB C and Fortran API has changed to support MATLAB
00066      variables with more than 2^32-1 elements. In the near future
00067      you will be required to update your code to utilize the
00068      \textcolor{keyword}{new} API. You can find more information about \textcolor{keyword}{this} at:
00069 
00070 http:\textcolor{comment}{//www.mathworks.com/help/matlab/matlab\_external/upgrading-mex-files-to-use-}
00071 64-bit-api.html.
00072 >>
00073 ```
00074 
00075 ### EXPERT PART 2. CONFIGURATION OF THE MEX COMPILER SETUP FILE.
00076 
00077 The previous step creates the following file:
00078 
00079 ```
00080 $(HOME)/.matlab/R2014a/mex\_C++\_glnxa64.xml
00081 ```
00082 
00083 The purpose of \textcolor{keyword}{this} section is to configure the MATLAB R2014a Mex compiler so
00084 that it can work with the latest C++ standard (C++11).
00085 
00086 Please execute the following changes on the aforementioned file (line numbers
00087 may differ):
00088 
00089 Lines 26 and 27:
00090 
00091 ```
00092 CMDLINE1=\textcolor{stringliteral}{"$CXX -std=c++11 -c $DEFINES $INCLUDE $CXXFLAGS $OPTIM $SRC -o $OBJ"}
00093 CMDLINE2=\textcolor{stringliteral}{"$LDXX -std=c++11 $LDFLAGS $LDTYPE $LINKOPTIM $LINKEXPORT $OBJS $CXXLIBS $LINKLIBS -o $EXE"}
00094 ```
00095 
00096 Line 33:
00097 
00098 ```
00099 CXXFLAGS=\textcolor{stringliteral}{"-std=c++11 -ansi -fexceptions -fPIC -fno-omit-frame-pointer -pthread"}
00100 ```
00101 
00102 Line 35 and 26:
00103 
00104 ```
00105 CXXOPTIMFLAGS=\textcolor{stringliteral}{"-std=c++11 -O -DNDEBUG"}
00106 CXXDEBUGFLAGS=\textcolor{stringliteral}{"-std=c++11 -g"}
00107 ```
00108 
00109 Line 38:
00110 
00111 ```
00112 LDXX=\textcolor{stringliteral}{"gfortran"}
00113 ```
00114 
00115 Line 59:
00116 
00117 ```
00118 CXXFLAGS=\textcolor{stringliteral}{"-std=c++11 -ansi -pthread"}
00119 ```
00120 
00121 ### NAIVE PART 1. CONFIGURATION OF THE MEX COMPILER.
00122 
00123 From your MATLAB prompt, type:
00124 
00125 ```
00126 >> mex -setup C++
00127 MEX configured to use \textcolor{stringliteral}{'g++'} for C++ language compilation.
00128 Warning: The MATLAB C and Fortran API has changed to support MATLAB
00129      variables with more than 2^32-1 elements. In the near future
00130      you will be required to update your code to utilize the
00131      new API. You can find more information about this at:
00132 
00133 http:\textcolor{comment}{//www.mathworks.com/help/matlab/matlab\_external/upgrading-mex-files-to-use-}
00134 64-bit-api.html.
00135 >>
00136 ```
00137 
00138 ### NAIVE PART 2. CONFIGURATION OF THE MEX COMPILER SETUP FILE.
00139 
00140 ```
00141 cd $HOME/.matlab/R2014a
00142 chmod +w mex\_C++\_glnxa64.xml
00143 patch < mex\_C++\_glnxa64.patch
00144 chmod -w mex\_C++\_glnxa64.xml
00145 ```
00146 Exit terminal, and restart MATLAB. You can use C++11 to create MEX files now!
00147 
00148 ### PART 3: CONFIGURATION OF THE MAKEFILE.
00149 
00150 The following steps are required the build and test the MTK. Please use the
00151 accompanying `Makefile.inc` file, which should provide a solid template to
00152 start with. The following command provides help on the options for make:
00153 
00154 ```
00155 $ make help
00156 -----
00157 Makefile for the MMTK.
00158 
00159 Options are:
00160 - \hyperlink{Div1D_8m_a4603254d9990f7140c024d51302d1a8f}{all}: builds he library, the tests, and examples.
00161 
00162 - gendoc: generates the documentation for the library.
00163 
00164 - clean: cleans ALL the generated files.
00165 -----
00166 ```
00167 
00168 ### PART 4. BUILD THE MMTK.
00169 
00170 From your shell, at the base folder of the MMTK, just type:
00171 
00172 ```
00173 make
00174 ```
00175 
00176 If successful you\textcolor{stringliteral}{'ll read:}
00177 \textcolor{stringliteral}{}
00178 \textcolor{stringliteral}{```}
00179 \textcolor{stringliteral}{----- Library created! Check in /home/ejspeiro/Dropbox/MTK/lib}
00180 \textcolor{stringliteral}{```}
00181 \textcolor{stringliteral}{}
00182 \textcolor{stringliteral}{And that is it.}
00183 \textcolor{stringliteral}{    \_\_\_\_\_\_\_\_\_\_\_\_\_\_\_\_\_\_\_\_\_\_\_\_\_\_\_\_\_\_\_\_\_\_\_\_\_\_\_\_\_\_\_\_\_\_\_\_\_\_\_\_\_\_\_\_\_\_\_\_\_\_\_\_\_\_}
00184 \textcolor{stringliteral}{}
00185 \textcolor{stringliteral}{## 4. Frequently Asked Questions}
00186 \textcolor{stringliteral}{}
00187 \textcolor{stringliteral}{Q: Why haven'}t you guys implemented GBS to build the library?
00188 A: I\textcolor{stringliteral}{'m on it as we speak! ;)}
00189 \textcolor{stringliteral}{}
00190 \textcolor{stringliteral}{Q: When will the other flavors be ready?}
00191 \textcolor{stringliteral}{A: Soon! I'}m working on getting help on developing those.
00192 
00193 Q: Is there any main reference when it comes to the theory on Mimetic Methods?
00194 A: Yes! Check: http:\textcolor{comment}{//www.csrc.sdsu.edu/mimetic-book}
00195 
00196 Q: Do I need to generate the documentation myself?
00197 A: You can if you want to... but if you DO NOT want to, just go to our website.
00198     \_\_\_\_\_\_\_\_\_\_\_\_\_\_\_\_\_\_\_\_\_\_\_\_\_\_\_\_\_\_\_\_\_\_\_\_\_\_\_\_\_\_\_\_\_\_\_\_\_\_\_\_\_\_\_\_\_\_\_\_\_\_\_\_\_\_
00199 
00200 ## 5. Contact, Support, and Credits
00201 
00202 The MTK is developed by researchers and adjuncts to the
00203 [Computational Science Research Center (CSRC)](http:\textcolor{comment}{//www.csrc.sdsu.edu/)}
00204 at [San Diego State University (SDSU)](http:\textcolor{comment}{//www.sdsu.edu/).}
00205 
00206 Developers are members of:
00207 
00208 1. Mimetic Numerical Methods Research and Development Group.
00209 2. Computational Geoscience Research and Development Group.
00210 3. Ocean Modeling Research and Development Group.
00211 
00212 Currently the developers are:
00213 
00214 - **Eduardo J. Sanchez, Ph.D. - esanchez at mail dot sdsu dot edu** - @ejspeiro
00215 - Jose E. Castillo, Ph.D. - jcastillo at mail dot sdsu dot edu
00216 - Guillermo F. Miranda, Ph.D. - unigrav at hotmail dot com
00217 - Christopher P. Paolini, Ph.D. - paolini at engineering dot sdsu dot edu
00218 - Angel Boada.
00219 - Johnny Corbino.
00220 - Raul Vargas-Navarro.
00221 
00222 Finally, please feel free to contact me with suggestions or corrections:
00223 
00224 **Eduardo J. Sanchez, Ph.D. - esanchez at mail dot sdsu dot edu** - @ejspeiro
00225 
00226 Thanks and happy coding!
\end{DoxyCode}
