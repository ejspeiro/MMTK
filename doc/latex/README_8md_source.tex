\hypertarget{README_8md_source}{\section{R\-E\-A\-D\-M\-E.\-md}
}

\begin{DoxyCode}
00001 \textcolor{preprocessor}{# MATLAB wrappers for the Mimetic Methods Toolkit (MMTK)}
00002 \textcolor{preprocessor}{}
00003 By: **Eduardo J. Sanchez, Ph.D. - esanchez at mail dot sdsu dot edu**
00004     \_\_\_\_\_\_\_\_\_\_\_\_\_\_\_\_\_\_\_\_\_\_\_\_\_\_\_\_\_\_\_\_\_\_\_\_\_\_\_\_\_\_\_\_\_\_\_\_\_\_\_\_\_\_\_\_\_\_\_\_\_\_\_\_\_\_
00005 
00006 \textcolor{preprocessor}{## 1. Description}
00007 \textcolor{preprocessor}{}
00008 We define numerical methods that are based on discretizations preserving the
00009 properties of their continuum counterparts to be **mimetic**.
00010 
00011 The **Mimetic Methods Toolkit (MTK)** is a C++ library \textcolor{keywordflow}{for} mimetic numerical
00012 methods. It is arranged as a set of classes \textcolor{keywordflow}{for} **mimetic quadratures**,
00013 **mimetic interpolation**, and **mimetic discretization** methods \textcolor{keywordflow}{for} the
00014 numerical solution of ordinary and partial differential equations.
00015 
00016 This collection of **MATLAB Wrappers \textcolor{keywordflow}{for} the MTK (MMTK)** allows developers to
00017 invoke the MTK from a MATLAB environment.
00018 
00019 Finally, a collection of grid visualization routines that is compatible with
00020 these wrappers can be found in:
00021 
00022 [MATLAB Visualizers \textcolor{keywordflow}{for} Uniform Staggered Grids](https:\textcolor{comment}{//github.com/ejspeiro/UniStgGrid-Visualizers)}
00023     \_\_\_\_\_\_\_\_\_\_\_\_\_\_\_\_\_\_\_\_\_\_\_\_\_\_\_\_\_\_\_\_\_\_\_\_\_\_\_\_\_\_\_\_\_\_\_\_\_\_\_\_\_\_\_\_\_\_\_\_\_\_\_\_\_\_
00024 
00025 \textcolor{preprocessor}{## 2. Dependencies}
00026 \textcolor{preprocessor}{}
00027 This README assumes \hyperlink{Div1D_8m_a4603254d9990f7140c024d51302d1a8f}{all} of these dependencies are installed in the following
00028 folder:
00029 
00030 ```
00031 $(HOME)/Libraries/
00032 ```
00033 
00034 In \textcolor{keyword}{this} version, the MTK optionally uses ATLAS-optimized BLAS and LAPACK
00035 routines \textcolor{keywordflow}{for} the \textcolor{keyword}{internal} computation on some of the layers. However, ATLAS
00036 requires both BLAS and LAPACK in order to create their optimized distributions.
00037 Therefore, the following dependencies tree arises:
00038 
00039 ### For Linux and OS X:
00040 
00041 1. MATLAB R2014a or greater - Available from: http:\textcolor{comment}{//www.mathworks.com/}
00042 
00043 1. MTK - Available from: http:\textcolor{comment}{//csrc.sdsu.edu/mtk/}
00044 
00045     \_\_\_\_\_\_\_\_\_\_\_\_\_\_\_\_\_\_\_\_\_\_\_\_\_\_\_\_\_\_\_\_\_\_\_\_\_\_\_\_\_\_\_\_\_\_\_\_\_\_\_\_\_\_\_\_\_\_\_\_\_\_\_\_\_\_
00046 
00047 ## 3. Installation
00048 
00049 The following instructions assume MATLAB R2014a or greater.
00050 
00051 You have two options, either follow the instructions given, **expert install**,
00052 or use the provided **patch file**, **naive install**.
00053 
00054 ### EXPERT PART 1. CONFIGURATION OF THE MEX COMPILER.
00055 
00056 From your MATLAB prompt, type:
00057 
00058 ```
00059 >> mex -setup C++
00060 MEX configured to use \textcolor{stringliteral}{'g++'} \textcolor{keywordflow}{for} C++ language compilation.
00061 Warning: The MATLAB C and Fortran API has changed to support MATLAB
00062      variables with more than 2^32-1 elements. In the near future
00063      you will be required to update your code to utilize the
00064      \textcolor{keyword}{new} API. You can find more information about \textcolor{keyword}{this} at:
00065 
00066 http:\textcolor{comment}{//www.mathworks.com/help/matlab/matlab\_external/upgrading-mex-files-to-use-}
00067 64-bit-api.html.
00068 >>
00069 ```
00070 
00071 ### EXPERT PART 2. CONFIGURATION OF THE MEX COMPILER SETUP FILE.
00072 
00073 The previous step creates the following file:
00074 
00075 ```
00076 $(HOME)/.matlab/R2014a/mex\_C++\_glnxa64.xml
00077 ```
00078 
00079 The purpose of \textcolor{keyword}{this} section is to configure the MATLAB R2014a Mex compiler so
00080 that it can work with the latest C++ standard (C++11).
00081 
00082 Please execute the following changes on the aforementioned file (line numbers
00083 may differ):
00084 
00085 Lines 26 and 27:
00086 
00087 ```
00088 CMDLINE1=\textcolor{stringliteral}{"$CXX -std=c++11 -c $DEFINES $INCLUDE $CXXFLAGS $OPTIM $SRC -o $OBJ"}
00089 CMDLINE2=\textcolor{stringliteral}{"$LDXX -std=c++11 $LDFLAGS $LDTYPE $LINKOPTIM $LINKEXPORT $OBJS $CXXLIBS $LINKLIBS -o $EXE"}
00090 ```
00091 
00092 Line 33:
00093 
00094 ```
00095 CXXFLAGS=\textcolor{stringliteral}{"-std=c++11 -ansi -fexceptions -fPIC -fno-omit-frame-pointer -pthread"}
00096 ```
00097 
00098 Line 35 and 26:
00099 
00100 ```
00101 CXXOPTIMFLAGS=\textcolor{stringliteral}{"-std=c++11 -O -DNDEBUG"}
00102 CXXDEBUGFLAGS=\textcolor{stringliteral}{"-std=c++11 -g"}
00103 ```
00104 
00105 Line 38:
00106 
00107 ```
00108 LDXX=\textcolor{stringliteral}{"gfortran"}
00109 ```
00110 
00111 Line 59:
00112 
00113 ```
00114 CXXFLAGS=\textcolor{stringliteral}{"-std=c++11 -ansi -pthread"}
00115 ```
00116 
00117 ### NAIVE PART 1. CONFIGURATION OF THE MEX COMPILER.
00118 
00119 From your MATLAB prompt, type:
00120 
00121 ```
00122 >> mex -setup C++
00123 MEX configured to use \textcolor{stringliteral}{'g++'} for C++ language compilation.
00124 Warning: The MATLAB C and Fortran API has changed to support MATLAB
00125      variables with more than 2^32-1 elements. In the near future
00126      you will be required to update your code to utilize the
00127      new API. You can find more information about this at:
00128 
00129 http:\textcolor{comment}{//www.mathworks.com/help/matlab/matlab\_external/upgrading-mex-files-to-use-}
00130 64-bit-api.html.
00131 >>
00132 ```
00133 
00134 ### NAIVE PART 2. CONFIGURATION OF THE MEX COMPILER SETUP FILE.
00135 
00136 ```
00137 cd $HOME/.matlab/R2014a
00138 chmod +w mex\_C++\_glnxa64.xml
00139 patch < mex\_C++\_glnxa64.patch
00140 chmod -w mex\_C++\_glnxa64.xml
00141 ```
00142 Exit terminal, and restart MATLAB. You can use C++11 to create MEX files now!
00143 
00144 ### PART 3: CONFIGURATION OF THE MAKEFILE.
00145 
00146 The following steps are required the build and test the MTK. Please use the
00147 accompanying `Makefile.inc` file, which should provide a solid template to
00148 start with. The following command provides help on the options for make:
00149 
00150 ```
00151 $ make help
00152 -----
00153 Makefile for the MMTK.
00154 
00155 Options are:
00156 - \hyperlink{Div1D_8m_a4603254d9990f7140c024d51302d1a8f}{all}: builds he library, the tests, and examples.
00157 
00158 - gendoc: generates the documentation for the library.
00159 
00160 - clean: cleans ALL the generated files.
00161 -----
00162 ```
00163 
00164 ### PART 4. BUILD THE MMTK.
00165 
00166 From your shell, at the base folder of the MMTK, just type:
00167 
00168 ```
00169 make
00170 ```
00171 
00172 If successful you\textcolor{stringliteral}{'ll read:}
00173 \textcolor{stringliteral}{}
00174 \textcolor{stringliteral}{```}
00175 \textcolor{stringliteral}{----- Library created! Check in /home/ejspeiro/Dropbox/MTK/lib}
00176 \textcolor{stringliteral}{```}
00177 \textcolor{stringliteral}{    \_\_\_\_\_\_\_\_\_\_\_\_\_\_\_\_\_\_\_\_\_\_\_\_\_\_\_\_\_\_\_\_\_\_\_\_\_\_\_\_\_\_\_\_\_\_\_\_\_\_\_\_\_\_\_\_\_\_\_\_\_\_\_\_\_\_}
00178 \textcolor{stringliteral}{}
00179 \textcolor{stringliteral}{## 4. Frequently Asked Questions}
00180 \textcolor{stringliteral}{}
00181 \textcolor{stringliteral}{Q: Why haven'}t you guys implemented GBS to build the library?
00182 A: I\textcolor{stringliteral}{'m on it as we speak! ;)}
00183 \textcolor{stringliteral}{}
00184 \textcolor{stringliteral}{Q: When will the other flavors be ready?}
00185 \textcolor{stringliteral}{A: Soon! I'}m working on getting help on developing those.
00186 
00187 Q: Is there any main reference when it comes to the theory on Mimetic Methods?
00188 A: Yes! Check: http:\textcolor{comment}{//www.csrc.sdsu.edu/mimetic-book}
00189 
00190 Q: Do I need to generate the documentation myself?
00191 A: You can if you want to... but if you DO NOT want to, just go to our website.
00192     \_\_\_\_\_\_\_\_\_\_\_\_\_\_\_\_\_\_\_\_\_\_\_\_\_\_\_\_\_\_\_\_\_\_\_\_\_\_\_\_\_\_\_\_\_\_\_\_\_\_\_\_\_\_\_\_\_\_\_\_\_\_\_\_\_\_
00193 
00194 ## 5. Contact, Support, and Credits
00195 
00196 The MTK is developed by researchers and adjuncts to the
00197 [Computational Science Research Center (CSRC)](http:\textcolor{comment}{//www.csrc.sdsu.edu/)}
00198 at [San Diego State University (SDSU)](http:\textcolor{comment}{//www.sdsu.edu/).}
00199 
00200 Developers are members of:
00201 
00202 1. Mimetic Numerical Methods Research and Development Group.
00203 2. Computational Geoscience Research and Development Group.
00204 3. Ocean Modeling Research and Development Group.
00205 
00206 Currently the developers are:
00207 
00208 - **Eduardo J. Sanchez, Ph.D. - esanchez at mail dot sdsu dot edu** - @ejspeiro
00209 - Jose E. Castillo, Ph.D. - jcastillo at mail dot sdsu dot edu
00210 - Guillermo F. Miranda, Ph.D. - unigrav at hotmail dot com
00211 - Christopher P. Paolini, Ph.D. - paolini at engineering dot sdsu dot edu
00212 - Angel Boada.
00213 - Johnny Corbino.
00214 - Raul Vargas-Navarro.
00215 
00216 Finally, please feel free to contact me with suggestions or corrections:
00217 
00218 **Eduardo J. Sanchez, Ph.D. - esanchez at mail dot sdsu dot edu** - @ejspeiro
00219 
00220 Thanks and happy coding!
\end{DoxyCode}
